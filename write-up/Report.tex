\documentclass{article}
\usepackage{graphicx}
\usepackage{float}
\usepackage[margin=2.5cm]{geometry}

\title
{
    Single-Image 3DGS Scene Reconstruction with Geometry-Aware Priors
}
\author
{
    Machine Visual Perception Course Project Report
}

\begin{document}
\maketitle

\section*{Information}
Authors: TEST
\newline
Group Number: TEST

\section{Chapter 1: Introduction and Motivation}
\subsection{Section 1.1: Introduction to the problem}
[Provide a thorough introduction to the problem and why it is important. Briefly explain what general techniques there are and how your project fits.]

\subsection{Section 1.2: Background and related work}
[Include a few very relevant related works and how your work relates to those, expanding on the previous section. We do not expect you to cover all previous works.]

\subsection{Section 1.3: Overview of the idea}
[Provide an overview stating why the idea of the project makes sense and what the main motivation is.]

\section{Chapter 2: Method}
\subsection{Section 2.1: Baseline algorithm}
[Explain the baseline architecture you used to build your algorithm on. You may reproduce figures from the original papers.]

\subsection{Section 2.2: Algorithm improvements}
[Explain what you implemented to improve over the baseline. You may include figures to explain the idea and logic. Focus on the ideas and not the implementation.]

\subsection{Section 2.3: Implementation details}
[Explain how you implemented the improvements. You may include code snippets with the corresponding explanations.]

\subsection{Section 2.3: Data pipelines}
[Explain your data format, how you consume the data in your algorithms, and data augmentation.]

\subsection{Section 2.4: Training procedures}
[Explain which framework and optimizers you use, how you implemented the training logic.]

\subsection{Section 2.5: Testing and validation procedures}
[Explain which framework you use, how you implemented the testing/ validation logic.]

\section{Chapter 3: Experiments and Evaluation}
\subsection{Section 3.1: Datasets}
[Explain the datasets utilized: what they contain, why they are utilized, assumptions, limitations, possible extensions.]

\subsection{Section 3.2: Training and testing results}
[Explain the training and testing results with graphs and elaborating on why they make sense, what could be improved.]

\subsection{Section 3.3: Qualitative results}
[Show in figures and explain visual results. Include different interesting cases covering different aspects/ limitations/ dataset diversity. If not converged, explain what we can expect once converged. Include any other didactic examples here.]

\subsection{[Optional] Section 3.4: Quantitative results}
[A table and associated explanations for quantitative results.]

\subsection{[Optional] Section 3.5: Comparison to state-of-the-art}
[Qualitative and/ or quantitative comparisons to one or more recent works, especially the baseline work.]

\section{Chapter 4: Conclusions and Future Directions}
\subsection{Section 4.1: Conclusions}
[Summarize what the project was about and the main conclusions.]

\subsection{Section 4.2: Discussion of limitations}
[Explain the limitations of your technique. You may want to refer to previous sections or show figures on the limitations.]

\subsection{Section 4.3: Future directions}
[State a few future directions for research and development. These typically follow from the discussion on limitations.]

\end{document}